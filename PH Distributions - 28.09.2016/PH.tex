\documentclass[envcountset]{beamer}
%
% Choose how your presentation looks.
%
% For more themes, color themes and font themes, see:
% http://deic.uab.es/~iblanes/beamer_gallery/index_by_theme.html


\mode<presentation>
{
  \usetheme{default}      % or try Darmstadt, Madrid, Warsaw, ...
  \usecolortheme{default} % or try albatross, beaver, crane, ...
  \usefonttheme{}  % or try serif, structurebold, ...
  \setbeamertemplate{navigation symbols}{}
  \setbeamertemplate{caption}[numbered]
  \setbeamertemplate{footline}[frame number]
} 
\usepackage{units}
\usepackage{mathtools}
\usepackage[english]{babel}
\usepackage[utf8x]{inputenc}
\usepackage{lmodern}
\usepackage{graphicx}
\usepackage{amssymb,amsmath,amsthm, amsmath}  % use mathematical symbols
\usepackage{bbold}
\usepackage{framed}
\usepackage{comment}
\usepackage[%
	font={small, sf},
	labelfont=bf,
	format=hang,
	format=plain,
	margin=0pt,
	width=0.8\textwidth,
]{caption}
\usepackage[list=true]{subcaption}

\usepackage{tikz}
\usetikzlibrary{matrix, arrows,positioning, fit, automata, calc}
\usepackage[absolute,overlay]{textpos}
\theoremstyle{definition}
\newtheorem{mydef}{Def.}[theorem]


\newenvironment{proenv}{\only{\setbeamercolor{local structure}{fg=green}}}{}



\title[PH Distributions]{Phase-type distributions}
\subtitle{}
\author{Fernando Zepeda}
\institute{TU Berlin - Lufthansa Systems}
\date{\today}

%source macro for the images

\setbeamercolor{framesource}{fg=gray}
\setbeamerfont{framesource}{size=\tiny}
\newcommand{\source}[1]{\begin{textblock*}{4cm}(8.7cm,8.6cm)

    \begin{beamercolorbox}[ht=0.5cm,right]{framesource}
        \usebeamerfont{framesource}\usebeamercolor[fg]{framesource} Source: {#1}
    \end{beamercolorbox}
\end{textblock*}}
%----------------------------------------

\begin{document}

\begin{frame}[noframenumbering, plain]
  \titlepage


\end{frame}

% Uncomment these lines for an automatically generated outline.
\begin{frame}{Outline}
  \tableofcontents
\end{frame}

\section{Introduction}

\begin{frame}{Introduction}

Some data do not follow necessarily the form of any standard distribution (unconventional forms):
	\begin{itemize}
	\item Probability distributions may posses any shape
	\item Rugged and multimodal data distributions
	\end{itemize}

\vspace{0.3cm}

The \textbf{phase-type (PH)} family of distributions have some useful properties in fitting this kind of data
	\begin{itemize}
	\item They are dense in the field of all positive-valued distributions $[ 0,\infty)$
	\item Thus, they can be used to approximate any positive-valued distribution Bladt. et al. (2003)
	\item Markovian properties make them analytically flexible
	\end{itemize}

\end{frame}
\begin{frame}{Motivation}

\textbf{Goal}: 
Estimate
\begin{itemize}
\item Buying probabilities (and ticketing probabilities)
\item Inter-arrival time of bookings
\end{itemize}

by fitting PH distributions to them given booking historical data 

\end{frame}

\section{Markov Chains}

\begin{frame}{Markov Chains}

Let $\mathbf{S}$ denote a countable set of states, and let $\{X(t)\}_{t\geq0}^{\infty}$ be a stochastic process with state space $\mathbf{S}$

	\begin{itemize}
	\item Let $n \in \mathbb{N}$ be the size of $\mathbf{S}$
	\end{itemize}

$\{X(t)\}_{t\geq0}^{\infty}$ is a continuous-time Markov chain \textbf{(CTMC)}, if it is characterized by the \textit{Markov property}: 
\\[10pt]
\centering
\begin{framed}
	$Prob(X(t_{k+1})=  x_{k+1} | X(t_k) = x_k, \ldots, X(t_0) = x_0) =$
	\\[4pt]
	$Prob(X(t_{k+1}) = x_{k+1} | X(t_k) = x_k)$
\end{framed}

for any $t_{k+1} \geq t_{k} \geq \ldots \geq t_1 \geq t_0 \geq 0 $ and $x_l \in \mathbf{S}$
\end{frame}

\begin{frame}{}

The transition between states $x$ are defined by the values $p_t(i,j)$ which are contained in a matrix with transition probabilities $\mathbf{P_t}$
\\[10pt]
\begin{framed}
The state probabilities at time $t$ are denoted by 
\\[5pt]
\centering
$p_t(j) = Prob(X(t) = j), j \in \mathbf{S}$ with $\sum_j p_t(j)=1$
\\[10pt]
$\pi(0) = [ p_0(1), p_0(2), \ldots, p_0(n)]$: initial probability vector 
\end{framed}
\centering
\flushleft
Random times between states transitions $V_1, V_2, \ldots, V_n$ are exponentially distributed random variables with parameter $\lambda(i)$ for state $i$
\flushleft
$V_i$: holding time in state $i$ with distribution function:
$$Prob(V_i \leq t) = 1 - e^{-\lambda(i)t}, t \geq 0$$
\end{frame}


\begin{frame}{}

The probabilistic behavior if the CTMC can be summarize in terms of its \underline{infinitesimal generator}: a $n\times n$ matrix $\mathbf{Q}$ with components:

$$
\mathbf{Q(i,j)} = 
\begin{cases}
  -\lambda(i)           & \text{if}\ i = j	,\\
  \phantom{-}\lambda(i,j) & \text{if}\ i \neq j .

\end{cases}
$$

\begin{framed}
\begin{itemize}
\item Since $\lambda(i) \geq 0$, it follows $\mathbf{Q}(i,i) \leq 0$: \textit{diagonal elements are non-positive}
\item Transition to $j$ feasible in state $i$ then $\mathbf{Q}(i,j) > 0$, otherwise $\mathbf{Q}(i,j) = 0$: \textit{non-diagonal elements must be non-negative}
\item $\sum_j\mathbf{Q}(i,j) = 0$
\end{itemize}
\end{framed}

\end{frame}

\begin{frame}{Absorbing Markov Chains}
\begin{figure}
\centering
\subcaptionbox[CTMC]{%
		\centering
		CTMC with transient states only%
      		\label{subfig:CTMC}%
}[
	0.40\textwidth
]
{
	\begin{tikzpicture}[node distance = 3cm, ->, >=latex, auto, every edge/.append style={thick}]
	\node[state] (1) {$1$};
	\node[state] (2)[right of =1]{$2$};
	\path (1) edge[bend left] node{$4$} (2)
	
	  	(2) edge[bend left] node{$2$} (1);
\end{tikzpicture}

}
\hspace{0.1\textwidth}
\subcaptionbox[Short Subcaption]{%
		\centering
		CTMC with one absorbing state%
      		\label{subfig:sublabel2}%
}[
	0.40\textwidth
]
{
\begin{tikzpicture}[node distance = 2.5cm, ->, >=latex, auto, every edge/.append style={thick}]
	\node[state](1) {$1$};
	\node[state](2)[right of =1]{$2$};
	\coordinate(Middle) at ($(1)!0.5!(2)$);
	\node[state, accepting](3)[below of =Middle]{$3$};
	
	\path  (1) edge[bend left] node{$4$} (2)
		  (2) edge[bend left] node{$2$} (1)
		  (2) edge[bend left] node{$\nicefrac{1}{2}$} (3)
		  (1) edge[bend right, left] node{$\nicefrac{1}{2}$} (3);
\end{tikzpicture}
}

\caption[Short Caption]{\centering Different Markov chains and their transition representation}
\label{fig:label}
\end{figure}
\end{frame}

\begin{frame}{Absorbing Markov Chains}
	\framesubtitle{Some definitions}

A state $j$ is reachable from another $i$ when we have 

$$p_t(i,j) = Prob(X(t+u) = j | X(u) = i) > 0$$
for some $t$

\begin{mydef} 
States $i$ and $j$ can communicate with each other if $i$ is reachable from $j$ and viceversa.
\end{mydef}
\vspace{5pt}
Let $\mathbf{C} \subset \mathbf{S}$. If all states in $\mathbf{C}$ communicate, we call it communicating set 

\begin{mydef}
A subset $\mathbf{C}$ of the state space $\mathbf{S}$ is said to be closed if $\mathbf{P}(i,j) = 0$ for any $i \in \mathbf{C}, j \not\in \mathbf{C}$.
\end{mydef}

\end{frame}

\begin{frame}
\hspace{5pt} If $\mathbf{C}$ consists of a single state, say $i$, then $i$ is said to be an absorbing state. 
\begin{framed}
It holds for $i$ that $\mathbf{P}(i,i) = 1$: \textit{a process can never leave a closed set after entering it}
\end{framed}
\begin{mydef}
A state $i \in \mathbf{S}$ is transient, if the probability of returning to $i$ after leaving it is less than 1.
\end{mydef}

\end{frame}

\section{Phase-type distributions}

\begin{frame}{PH distributions and absorbing Markov chains}

If every state in a Markov chain is either absorbing or transient, then the Markov chain is called \textit{absorbing Markov chain}

\begin{framed}
	Particularly interesting case: absorbing Markov chains with one single absorbing state
\end{framed}

\underline{Assumption}: state space $\mathbf{S}$ of the continuous time absorbing Markov process $\{X(t)\}_{t\geq0}^{\infty}$ is finite and contains the set of \textbf{transient} states $\mathbf{S}_T = \{1, \ldots, n\}$ and a single absorbing state $n+1$

\end{frame}

\begin{frame}[fragile]
\Large

\[ \mathbf{Q} = 
\begin{tikzpicture}[baseline=-0.5]

\matrix[matrix of math nodes, left delimiter = (, right delimiter = ) ](m){
\mathbf{D_0} & \mathbf{d_1}\\
\mathbf{0} & 0\\
};
\node<2>[label={\tiny $n\times 1$},fit=(m-1-2), rounded corners, draw=blue, scale=.6]{};
\node<2>[label={\tiny $n\times n$},fit=(m-1-1), draw=red, scale=.6]{};
\node<2>[label=below:{\tiny $1\times n$},fit=(m-2-1), draw=green, scale=.6]{};
\node<2>[label=below:{\tiny $1\times 1$},fit=(m-2-2), draw=violet, scale=.6]{};
\end{tikzpicture}
\]

\normalsize
is the generator matrix, where:

\begin{itemize}
\item $\mathbf{D_0}$: describes transitions between transient states
\item $\mathbf{d_1}$: transition from transient states to the absorbing state
\item row vector $\mathbf{0}$: since no transition from the absorbing state to the transient states
\end{itemize}

\end{frame}

\begin{frame}
Since the states are transient, implies:

$$\lim_{t\to\infty} Prob(X(t)<n+1)=0,$$

i.e. the absorption occurs with probability 1.

\vspace{5pt}

$\mathbf{D_0}$ is invertible and matrix $(-\mathbf{D_0})^{-1}$ is the fundamental matrix of the absorbing CTMC 
\begin{framed}
\begin{block}{Def. Phase-type distribution (PHD)}

A PHD is defined as the distribution of the lifetime $X$ or the time to absorption from the set of transient states $\mathbf{S_T}$ of an absorbing continuous Markov process $\{X(t)\}_{t\geq0}^{\infty}$

\end{block}
\end{framed}
\end{frame}

\begin{frame}{Phase Type Distributions (PHD)}

The Markov process starts in an arbitrary state from $\mathbf{S}=\mathbf{S_T} \cup \mathbf{S_A}$, so the vector 
$$\pi=[\pi(1),\ldots,\pi(n)]$$

represents the inital probabilites for transient states and $\pi(n+1)$ gives the probability for a direct start in the absorbing (assumption: $\pi(n+1)=0$)
\end{frame}

\begin{frame}{Phase Type Distributions (PHD)}
\framesubtitle{Simple example}

$\mathbf{S}=\mathbf{S_T} \cup \mathbf{S_A} = \{1,2\} \cup \{3\}$

\begin{minipage}[c]{0.45\textwidth}
	\begin{minipage}[c]{1.\textwidth}
	\begin{equation*}
	\mathbf{Q} =
	\left(
	\begin{array}{cc|c}
	-4.5 & 4 & 0.5 \\
	2 & -2.5 &0.5\\
	\hline
	0 &0 &0 \\
	\end{array}
	\right)
	%%
	\end{equation*}
	\end{minipage}
	\vfill
	\begin{minipage}{1.\textwidth}
	\flushright
	\begin{equation*}
	\pi=[\frac{1}{2},  \frac{1}{2}]
	\end{equation*}
	\end{minipage}
\end{minipage}
\hfill
\begin{minipage}[c]{0.45\textwidth}
\begin{tikzpicture}[node distance = 2.5cm, ->, >=latex, auto, every edge/.append style={thick}]
	\node[state, initial, initial text={$\nicefrac{1}{2}$}, initial where=above](1) {$1$};
	\node[state, initial, initial text={$\nicefrac{1}{2}$},initial where=above](2)[right of =1]{$2$};
	\coordinate(Middle) at ($(1)!0.5!(2)$);
	\node[state, accepting](3)[below of =Middle]{$3$};
	
	\path  (1) edge[bend left] node{$4$} (2)
		  (2) edge[bend left] node{$2$} (1)
		  (2) edge[bend left] node{$\nicefrac{1}{2}$} (3)
		  (1) edge[bend right, left] node{$\nicefrac{1}{2}$} (3);
\end{tikzpicture}
\end{minipage}

\end{frame}

\begin{frame}{PHD Representation}

We can say now that the random variable $X$ describing the time until absorption is of phase-type with the vector-matrix tuple representation:

$$(\pi, \mathbf{D_0})$$

since the vectors $\mathbf{d_1}$ and the value $\pi(n+1)$ are implicitly given by the matrix $\mathbf{D_0}$

\vspace{10pt}
The size of the PHD is the size of the sub-generator matrix $\mathbf{D_0}$, $n$

\end{frame}

\begin{frame}{Analytical Properties of PHD}
$$F(x) = \mathbb{1} - \pi e^{\mathbf{D_0}x}\mathbb{1}, \text{ for }  x \geq 0 \text{ (c.d.f.)}$$

$$f(x) = \pi e^{\mathbf{D_0}x}\mathbf{d_1}, \text{ for }  x \geq 0 \text{   (p.d.f.)}$$

$$\mu_i=E[X^i]=i ! \pi \mathbf{(-D_0^{-1})}^i \mathbb{1}, \text{ PHD $i$th moment } $$

\vspace{10pt}

Note 1: $\mathbf{M} = \mathbf{-D_0^{-1}}$ is the moment matrix
$$\rightarrow \mu_i=E[X^i]=i ! \pi \mathbf{\mathbf{M}}^i \mathbb{1}$$

Note 2: $\mathbf{-D_0^{-1}}(i,j)=\mathbf{M}(i,j)$ is the expected total time spent in phase $j$ before absorption, given that initial phase is $i$

\end{frame}

\section{Classes}
\begin{frame}{PH Classes}

Based on the structure of the underlying Markov chain several classes of PH distributions can be distinguished

	\begin{itemize}
	\item Structure of PH representation often has an impact on its application
	\item Some structures allow more efficient solutions
	\end{itemize}

Most important distinction: 

	\begin{enumerate}
	\item\textcolor<2,3>{blue}{Acyclic phase-type distributions (APH)}
		\onslide<3>{
		\begin{itemize}
		\item Hyper-Erlang distributions 
		\item Hyper-Exponential distributions
		\end{itemize}}
	\item General phase-type distributions
	\end{enumerate}


\end{frame}
\begin{frame}{Hyper-Erlang distributions (HEr PHD)}

\begin{figure}
\centering
\begin{tikzpicture}[minimum size=1pt, node distance = 2cm, ->, >=latex, auto, every edge/.append style={thick}]
	\node[state, initial, initial text={$\beta_1$}](1) {$$};
	\node[state](2)[right of =1]{$$};
	\node[state](3)[right of =2]{$$};
	\node[state, initial, initial text={$\beta_2$}](4)[below of =1]{$$};
	\node[state](5)[right of =4]{$$};
	\coordinate(medio) at ($(5)!0.9!(3)$);
		\coordinate(Middle) at ($(1)!0.5!(4)$);
	\node[state, accepting](6)[right of =medio]{$$};
	
	\path  (1) edge node{$\lambda_1$} (2)
		  (2) edge node{$\lambda_1$} (3)
		  (3) edge[above] node{$\lambda_1$} (6)
		  (4) edge[below] node{$\lambda_2$} (5)
		  (5) edge[below] node{$\lambda_2$} (6);
\end{tikzpicture}
\caption{Hyper-Erlang distribution with two branches $(\beta, m, \mathbf{b}, \lambda)$ = $([\beta_1, \beta_2], 2, [3,2], [\lambda_1, \lambda_2])$ }
\end{figure}

Size of the Hyper-Erlang distribution: $n = \mathbf{b}\mathbb{1}$

\vspace{5pt}
\underline{Special case}: 
$(\beta, m, \mathbf{b}, \lambda) = ([\beta_1=1], 1, \mathbf{b}, \lambda_1)$ 
$\rightarrow$ Erlang distr. 

\end{frame}
\begin{frame}{Hyper-Exponential distributions (HEx PHD)}
\begin{figure}
\centering
\begin{tikzpicture}[minimum size=1pt, node distance = 2cm, ->, >=latex, auto, every edge/.append style={thick}]
	\node[state, initial, initial text={$\beta_1$},  initial where=below](1) {$$};
	\node[state, initial, initial text={$\beta_2$}, initial where=below](2)[right of =1]{$$};
	\node[state, initial, initial text={$\beta_3$}, initial where=below](3)[right of =2]{$$};
	\node[state, initial, initial text={$\beta_4$}, initial where=below](4)[right of =3]{$$};
	\node[state, accepting](5)[right of =4]{$$};
	
	\path  (1) edge[bend left] node{$\lambda_1$} (5)
		  (2) edge[bend left, right] node[above,pos=0.1] {$\lambda_2$} (5)
		  (3) edge[bend right] node[above, pos=0.2]{$\lambda_3$} (5)
		  (4) edge node{$\lambda_4$} (5);
\end{tikzpicture}
\caption{Hyper-Exponential distribution with four branches $(\beta, n, \lambda)$ = $([\beta_1, \beta_2, \beta_3, \beta_4], 4, [\lambda_1, \lambda_2, \lambda_3, \lambda_4])$}
\end{figure}

\underline{Special case}: $(\beta, n, \lambda) = ([\beta=1], 1, \lambda_1) \rightarrow$ Exponential distr.

\vspace{5pt}
Note: HEx are subclass of HEr when $\mathbf{b}=\mathbb{1} \text{ and } \mathbf{m}=n$ 

\end{frame}

\section{Canonical Representations}

\begin{frame}{Canonical Representation for APH distributions}
\framesubtitle{CF-1}

\begin{mydef}
The Canonical Form 1 (CF-1 form) is a bi-diagonal representation $(\beta, \Lambda)$, where $\beta$ initial probabilities and rates $\lambda_i$ in $\Lambda$ are in increasing order: $\lambda_1 \leq \lambda_2 \leq \ldots \leq \lambda_n$
\end{mydef}


\begin{figure}
\centering
\begin{tikzpicture}[minimum size=1pt, node distance = 2cm, ->, >=latex, auto, every edge/.append style={thick}]
	\node[state, initial, initial text={$\beta_1$},  initial where=below](1) {$$};
	\node[state, initial, initial text={$\beta_2$}, initial where=below](2)[right of =1]{$$};
	\node[state, initial, initial text={$\beta_3$}, initial where=below](3)[right of =2]{$$};
	\node[state, initial, initial text={$\beta_4$}, initial where=below](4)[right of =3]{$$};
	\node[state, accepting](5)[right of =4]{$$};
	
	\path  (1) edge node{$\lambda_1$} (2)
		  (2) edge node{$\lambda_2$} (3)
		  (3) edge node{$\lambda_3$} (4)
		  (4) edge node{$\lambda_4$} (5);
\end{tikzpicture}
\caption{APH distribution in CF-1 form}
\end{figure}

\end{frame}

\begin{frame}{Canonical Representation for APH distributions}

\begin{framed}
Every acyclic phase type distribution (APH) with a Markovian representation of size $n$ has an unique CF-1 representation of the same size.
\end{framed}
\begin{itemize}
\item A general APH has $n-1$ initial probabilities and $n^2$ entries in the sub-generator matrix $\mathbf{D_0}$, i.e. the number of parameters is $n^2+n-1$

\item In the CF-1 form $\mathbf{D_0}$ is an upper bi-diagonal matrix with $\lambda_{i,i}=-\lambda_{i+1,i}$, therefore the CF-1 form has $2n-1$ parameters
\end{itemize}
\end{frame}

\begin{frame}{PH-Fitting Tools}
Two general classes of algorithms
\begin{itemize}
\item Analytical: direct computation of parameters
	\begin{itemize}
	\item Moment Matching: $\mu_1, \mu_2, \mu_3$ 
	\end{itemize}
\item Statistical: maximum likelihood method (iterative process)
	\begin{itemize}
	\item G-FIT: Hyper-Erlang distributions
	\item PhFit: Acyclic PHD CF-1 form  
	\end{itemize}
\end{itemize}

\end{frame}

\begin{frame}{Summary}

\begin{itemize}
\item PHD can be used to approximate any positive-valued distribution
\item They represent the distribution of the time to absorption of an absorbing continuous time Markov process $\{X(t)\}_{t\geq0}^{\infty}$ 
\item They are defined by the tuple $(\pi, \mathbf{D_0})$
\item Two big families: Hyper-Erlang and Hyper-Exponential distributions
\item Acyclic phase-type distributions deserve an special attention
\item They can be reduced in a canonical form in order to estimate their parameters $(2n-1)$
\item Fitting techniques
\end{itemize}

\end{frame}

\begin{frame}[noframenumbering, plain]{References}

\begin{thebibliography}{2}
\footnotesize{
\setbeamertemplate{bibliography item}[book]
\bibitem{A} Bucholtz, Kriege \& Felko (2014). Input Modeling with Phase-Type Distributions and Markov Models: Theory and Applications 
\setbeamertemplate{bibliography item}[article]
\bibitem{A} Cumani, A (1982). On the canonical representation of homogeneous Markov processes modelling failure-time distributions microelectronics and reliability 22. 582-602.
\setbeamertemplate{bibliography item}[article]
\bibitem{B} Reinicke, P \& Horváth G. (2012). Phase-type distributions for realistic modeling in discrete-event simulation. \textit{In Proceedings of the 5th International ICST Conference on Simulation Tools and Techniques} (pp.283-290). ICST (Institute for Computer Sciences, Social-Informatics and Telecommunications Engineering).}

\end{thebibliography}

\end{frame}




\end{document}

              